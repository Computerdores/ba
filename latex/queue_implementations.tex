\section{Queue Implementations}
\begin{figure}
    \begin{minted}{cpp}
        template <typename T>
        class Queue {
          public:
            using Item = T;

            virtual bool enqueue(T item) = 0;

            virtual std::optional<T> dequeue() = 0;

            virtual ~Queue() = default;
        };
    \end{minted}
    \caption{The \texttt{Queue<T>} interface.}
    \label{fig:queue-interface}
\end{figure}

\begin{itemize}
    \item all of the queues needed to be either ported to cpp, adapted to share the same interface, or
        reimplemented because no source code was available (For interface see \autoref{fig:queue-interface})
    \item this section is about the considerations that were made in the process
\end{itemize}

\todo{TODO: update the 'as of now' dates later}
On 2025-09-04, I reached out via E-Mail to the authors of the MCRingBuffer and FastForward papers seeking
source code for the original implementations or feedback on my implementations of their queues.
One of the authors of the FastForward paper responded and was able to provide me with a partial
implementation of the FastForward queue which was missing the slip adjustment code.
The provided code was very similar to my implementation, suggesting that that part of my implementation is correct.
I have not received an answer from the authors of the MCRingBuffer paper as of 2025-09-17, however, my
original message could not be delivered to Tian Bu and Girish Chandranmenon since their E-Mail addresses,
stated in the MCRingBuffer paper, are not valid anymore.

I reached out via E-Mail to Yangfeng Tian and Xiong Fu, who are authors of the EQueue paper, and Junchang
Wang, who is an author of the B-Queue and EQueue papers, on 2025-05-28 with some questions about the EQueue
paper and source code
\footnote{About a week after reaching out, I noticed that the EQueue repository on GitHub ceased to be available.
For preservation purposes, I have reuploaded the repository to GitHub (See \href{https://github.com/Computerdores/equeue}{Computerdores/equeue}).
For the same purpose, I have also forked a GitHub reupload of the now unavailable, original B-Queue repository (See \href{https://github.com/Computerdores/b-queue}{Computerdores/b-queue}).}.
I have not received an answer to this E-Mail as of 2025-09-17 and for this reason not reached out separately
about feedback on my implementation.

\subsection{BQueue}
\begin{itemize}
    \item originally reimplemented from paper because source was unavailable
    \item full implementation, including self-adaptive backtracking and prod/con batching
    \item source has since been found elsewhere and original version showed no measurable performance difference
    \item two small divergences from the pseudo code in the paper are present to fix bugs
    \item one check added to prevent tail from overtaking batch tail
    \item one check added to prevent head from overtaking tail in certain circumstances
        \footnote{Not necessary for benchmarks anymore; left in because more robust and there seemed to be no
        harm in it}
\end{itemize}

\subsection{EQueue}
\begin{itemize}
    \item ported to cpp with small differences because there were differences between the released source
        code and the paper
    \item implementation in released source code is also not entirely self-contained
    \item I tried to reconcile the differences apropriately
    \item added check that used size doesn't exceed buffer size (present in the released implementation, but
        not in the paper)
    \item \texttt{\_enqueue\_detect\_batching\_size} differs between paper and their impl, I chose closer to their impl.
    \item \texttt{traffic\_empty} should be increased in \texttt{dequeue} according to paper
        \begin{itemize}
            \item their code does it outside of dequeue method
            \item I did inside dequeue method to have self-contained impl
        \end{itemize}
    \item paper doesn't increase \texttt{traffic\_full} in batched version of \texttt{enqueue}
        \begin{itemize}
            \item this doesn't make any sense, because it breaks the core mechanism of the queue
            \item their code does increase it, but outside of queue impl
            \item I once again did in in the queue impl to stay self-contained
        \end{itemize}
    \item proposed LT-CAS is not implemented since their source code only demonstrates it, but doesn't use it
        in the queue
\end{itemize}

\subsection{MCRingBuffer}
\begin{itemize}
    \item no publicly available source code
    \item padding between shared, consumer local, producer local, and read-only variables
    \item volatile markings for read, write, and buffer variables
    \item paper specifies busy wait when enqueue/dequeue don't immediately succeed; to conform with interface
        this was changed to return failure instead
    \item deadlocks under certain conditions (possible mitigations outlined in bqueue paper, not pursued here)
\end{itemize}

\subsection{Fast Flow Queue}
\begin{itemize}
    \item adapted to fit interface
    \item by default has prepare/commit interface
    \item added wrapper methods that implement a simple enqueue/dequeue according to the general interface used here
\end{itemize}

\subsection{Fast Forward Queue}
\begin{itemize}
    \item no publicly available source code
    \item padding between head, tail, and buffer ptr to put them on different cache lines
    \item volatile references to allow compiler to optimise accesses to head from enqueue and tail from
        dequeue normally, but correctly see accesses to tail from enqueue and head from dequeue as volatile
    \item implementation of distance function was not specified in paper, difference between head and tail
        used (head will be accessed from dequeue; otherwise only from enqueue)
\end{itemize}

\subsection{Lamport Queue}
\begin{itemize}
    \item benchmarked in FastForward paper, but source code not published => reimplemented
    \item padding between head, tail and read-only variables
    \item volatile references as before
    \item no busy wait like specified in Lamport paper
\end{itemize}
