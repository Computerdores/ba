\section{Queue Implementations}
All of the queues I will benchmark in this work, needed to either be ported to cpp, adapted to share the same
interface (see \autoref{fig:queue-interface}), or reimplemented because no source code was available.
In the first subsection, I will explain how I reached out to authors of those queues for feedback on my implementations.
In the following subsections, I will outline the considerations that were made for each queue one at a time.

\subsection{Feedback}
\todo{TODO FINAL: update the 'as of now' dates later}
On 2025-09-04, I reached out via E-Mail to the authors of the MCRingBuffer and FastForward papers seeking
source code for the original implementations or feedback on my implementations of their queues.
One of the authors of the FastForward paper responded and was able to provide me with a partial
implementation of the FastForward queue which was missing the slip adjustment code.
The provided code was very similar to my implementation, suggesting that that part of my implementation is correct.
I have not received an answer from the authors of the MCRingBuffer paper as of 2025-09-23, however, my
original message could not be delivered to Tian Bu and Girish Chandranmenon since their E-Mail addresses,
stated in the MCRingBuffer paper, are not valid anymore.

I reached out via E-Mail to Yangfeng Tian and Xiong Fu, who are authors of the EQueue paper, and Junchang
Wang, who is an author of the B-Queue and EQueue papers, on 2025-05-28 with some questions about the EQueue
paper and source code\footnote{About a week after reaching out,
    I noticed that the EQueue repository on GitHub ceased to be available.
    For preservation purposes, I have reuploaded the repository to GitHub (See
    \href{https://github.com/Computerdores/equeue}{Computerdores/equeue}).
    For the same reason, I have also forked a GitHub reupload of the, now unavailable, original B-Queue
    repository (See \href{https://github.com/Computerdores/b-queue}{Computerdores/b-queue}).
\label{equeue-bqueue-links}}.
I have not received an answer to this E-Mail as of 2025-09-23 and for this reason not reached out separately
about feedback on my implementation.

\begin{listing}
\begin{minted}{cpp}
template <typename T>
class Queue {
  public:
    using Item = T;

    // non-blocking on failure
    virtual bool enqueue(T item) = 0;

    // non-blocking on failure
    virtual std::optional<T> dequeue() = 0;

    virtual ~Queue() = default;
};
\end{minted}
\caption{The \texttt{Queue<T>} interface.}
\label{fig:queue-interface}
\end{listing}


\todo{TODO FINAL: Add link to repository for code of each queue}

\subsection{BQueue}
This queue was initally reimplemented, because the original source code repository linked in the B-Queue
paper had become unavailable\cite{B-Queue}.
As such a full B-Queue implemention, including self-adaptive backtracking and producer and consumer batching, was made.
I have since found a copy of the original repository and validated that my implementation shows the same
performance characteristics in my benchmarks.
The implementation has two small divergences from the pseudo code in the paper, which are present to fix bugs.
First, a check was added to prevent the tail from overtaking the batch tail.
Second, another check was added to prevent the head from overtaking the tail in certain circumstances\footnote{
    This check does not appear to be necessary for the benchmarks anymore, however, I did not remove it,
because it makes the implementation more robust and I did not observe any performance difference due to it.}.

\subsection{EQueue}
EQueue was ported to C++ with small differences, because the released source code and the paper differed\cite{EQueue}.
Furthermore, their queue implementation is also not entirely self contained (see traffic counters).
I tried to reconcile the differences between the paper and their code appropriately.

First, I added a check to make sure the used portion of the buffer does not exceed the buffer in size, which
is also present in their code.
Secondly, the structure of \texttt{\_enqueue\_detect\_batching\_size} differs between the EQueue paper and
their implementation; my version of it is closer to their implementation than to the paper.
Thirdly, according to the paper, the \texttt{traffic\_empty} counter should be increased in \texttt{dequeue},
however their implementation does this outside of the queue implementation.
To keep my implementation self contained, I moved this into the \texttt{dequeue} method, as specified by the paper.
Fourthly, the EQueue paper's pseudo code does not increment \texttt{traffic\_full} in the batched version of
\texttt{enqueue}.
In my opinion this does not make a lot of sense, since the core mechanism to resize the used portion of the
buffer relies on this being incremented.
Furthermore, their implementation also increments it, but does so outside of the queue implementation.
For my implementation, I increment it as part of the \texttt{enqueue} method, similarly to the
\texttt{traffic\_empty} counter.
Lastly, I did not implement their LT-CAS primitive, since their implementation does not make use of it as
part of the queue either (it only contains a demonstration of the primitive).

\subsection{MCRingBuffer}
The queue from the MCRingBuffer paper was implemented from scratch, as there does not seem to be any publicly
available source code\cite{MCRingBuffer}.
I made sure to add padding between shared, consumer local, producer local, and read only variables, to
prevent false sharing by making sure these blocks of variables are on different cache lines.
I also added volatile markings for the variables \texttt{read}, \texttt{write}, and \texttt{buffer}.
The last change I made is that when enqueue or dequeue don't immediately succeed, instead of waiting as
specified in the paper, the operation instead fails, to conform with the interface\cite{MCRingBuffer}.
It is also worth mentioning that this queue deadlocks under certain conditions, however, fixing this issue
was not pursued here, for the same reasons also outlined in the B-Queue paper\cite{B-Queue}.

\subsection{FastFlow Queue}
FastFlow Queue was largely taken as is and only extended to conform with the general queue interface\cite{FastFlowGithub}.
This was done since FastFlow Queue has a prepare + commit interface for enqueue and dequeue.
To fix this interface discrepancy, wrapper methods were implemented, which make use of the prepare + commit
interface to implement the typical enqueue and dequeue semantics.

\subsection{FastForward Queue}
Like MCRingBuffer, FastForward queue was implemented from scratch, because I was not able to find source code for it\cite{FastForward}.
Also like MCRingBuffer, padding was added to make sure that the \texttt{head}, \texttt{tail}, and
\texttt{buffer} variables are on different cache lines, to prevent false sharing.
Furthermore, volatile markings and volatile references were added to prevent optimisation errors by the compiler.
Lastly, the implementation of the distance function was not specified in the paper and was implemented via
access to the, otherwise thread local, \texttt{head} and \texttt{tail} variables.

\subsection{Lamport Queue}
This queue was also reimplemented due to no source code being available, which had seemingly also been done
for the FastForward paper.
Similar to the other queues, padding was added between \texttt{head}, \texttt{tail}, and the read only
variables and volatile references were added as well.
It is also worth noting that, as with MCRingBuffer, instead of waiting when an operation does not succeed
immediately like the paper specifies, the queue operations fail instead\cite{Lamport}.
