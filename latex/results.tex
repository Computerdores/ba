\subsection{Results}
It can generally be noted that the jitter on the TX side does not have an impact on any of the results
(See \cref{fig:results-bursty-65k-16k,fig:results-bursty-65k-2k,fig:results-basic-65k}).
Jitter having a significant impact can be a problem where the precise timing by the benchmark causes
significant changes in performance.
This was also observed at times during the development of the benchmarks, but disappeared after applying bug fixes.

Furthermore, regarding comparisons to the EQueue paper, three things are of note.
First, the measurements from the EQueue paper include parts of their measurement code \todo{TODO REVIEW: cite
    source code
here}, my measurements do not.
However, it is not clear to me how significant the impact on their measurements would be.
Second, I suspect that the measurements with different burst sizes from the EQueue paper were done in their
cross-cpu configuration,
which would be a significant difference to my measurements, but the paper does not specify whether or not
this is the case.
I suspect this, because the values for burst size 2048 from the graph match the cross-cpu values from their
table better than the same-cpu values.
Third, the values for the different burst sizes are not directly stated and were therefore discerned from the
graph included in the paper.

In the following subsubsections, I analyse the results from the different benchmarks.
I first address the two bursty benchmarks and compare the results to the results from the EQueue paper, as
they also used bursty enqueue timing.
Lastly, for the Basic benchmark I compare my results to those of the B-Queue and FastForward papers as this
benchmark is, out of mine, most similar to those.

\providecommand{\imgpathprefix}{}
\providecommand{\imgcaption}{}
\providecommand{\imglabel}{}

\renewcommand{\imgpathprefix}{../../plots/bdf4473_2025-09-07_bursty_}
\renewcommand{\imgcaption}{Bursty 65k-16k results}
\renewcommand{\imglabel}{fig:results-bursty-65k-16k}

\begin{figure}
    \centering
    \begin{subfigure}{0.45\textwidth}
        \centering
        \includegraphics[width=\linewidth]{\imgpathprefix False_False}
        \caption{}
        \label{\imglabel -a}
    \end{subfigure}
    \hfill
    \begin{subfigure}{0.45\textwidth}
        \centering
        \includegraphics[width=\linewidth]{\imgpathprefix False_True}
        \caption{}
        \label{\imglabel -b}
    \end{subfigure}

    \begin{subfigure}{0.45\textwidth}
        \centering
        \includegraphics[width=\linewidth]{\imgpathprefix True_False}
        \caption{}
        \label{\imglabel -c}
    \end{subfigure}
    \hfill
    \begin{subfigure}{0.45\textwidth}
        \centering
        \includegraphics[width=\linewidth]{\imgpathprefix True_True}
        \caption{}
        \label{\imglabel -d}
    \end{subfigure}
    \caption{\imgcaption}
    \label{\imglabel}
\end{figure}


\providecommand{\imgpathprefix}{}
\providecommand{\imgcaption}{}
\providecommand{\imglabel}{}

\renewcommand{\imgpathprefix}{../../plots/7bca0f3_2025-09-13_eq2k_bursty_}
\renewcommand{\imgcaption}{Results for the Bursty 65k-2k benchmark, with error bars showing the 95\%
confidence interval.}
\renewcommand{\imglabel}{fig:results-bursty-65k-2k}

\begin{figure}
    \centering
    \begin{subfigure}{0.45\textwidth}
        \centering
        \includegraphics[width=\linewidth]{\imgpathprefix False_False}
        \caption{}
        \label{\imglabel -a}
    \end{subfigure}
    \hfill
    \begin{subfigure}{0.45\textwidth}
        \centering
        \includegraphics[width=\linewidth]{\imgpathprefix False_True}
        \caption{}
        \label{\imglabel -b}
    \end{subfigure}

    \begin{subfigure}{0.45\textwidth}
        \centering
        \includegraphics[width=\linewidth]{\imgpathprefix True_False}
        \caption{}
        \label{\imglabel -c}
    \end{subfigure}
    \hfill
    \begin{subfigure}{0.45\textwidth}
        \centering
        \includegraphics[width=\linewidth]{\imgpathprefix True_True}
        \caption{}
        \label{\imglabel -d}
    \end{subfigure}
    \caption{\imgcaption}
    \label{\imglabel}
\end{figure}


\providecommand{\imgpathprefix}{}
\providecommand{\imgcaption}{}
\providecommand{\imglabel}{}

\renewcommand{\imgpathprefix}{../../plots/7bca0f3_2025-09-13_eq2k_basic_}
\renewcommand{\imgcaption}{Results for the Basic 65k benchmark, with error bars showing the 95\% confidence interval.}
\renewcommand{\imglabel}{fig:results-basic-65k}

\begin{figure}
    \centering
    \begin{subfigure}{0.45\textwidth}
        \centering
        \includegraphics[width=\linewidth]{\imgpathprefix False_False}
        \caption{}
        \label{\imglabel -a}
    \end{subfigure}
    \hfill
    \begin{subfigure}{0.45\textwidth}
        \centering
        \includegraphics[width=\linewidth]{\imgpathprefix False_True}
        \caption{}
        \label{\imglabel -b}
    \end{subfigure}

    \begin{subfigure}{0.45\textwidth}
        \centering
        \includegraphics[width=\linewidth]{\imgpathprefix True_False}
        \caption{}
        \label{\imglabel -c}
    \end{subfigure}
    \hfill
    \begin{subfigure}{0.45\textwidth}
        \centering
        \includegraphics[width=\linewidth]{\imgpathprefix True_True}
        \caption{}
        \label{\imglabel -d}
    \end{subfigure}
    \caption{\imgcaption}
    \label{\imglabel}
\end{figure}



\todo{TODO FINAL: My numbers here are read from figures, replace them with the accurate values}

\subsubsection{Bursty 65k-16k}
As mentioned in \autoref{sec:framework}, the components give two options to be varied with each benchmark:
jitter and whether failed operations are included in the measurements.
The four variations of this benchmark can be seen in \autoref{fig:results-bursty-65k-16k}.
The variation in \autoref{fig:results-bursty-65k-16k-b} without jitter and which includes failed operations
in the measurements is closest to the burst size $16,384$ test from the EQueue paper\cite{EQueue}.

According to the EQueue paper, EQueue reportedly took $\approx 60$ cycles per operation, which is
$\approx \ns{36}$.
My measurements show a time per operation for EQueue of $\approx \ns{24}$ for enqueue and
$\approx \ns{26}$ for dequeue.
These results suggest that my EQueue implementation does not exhibit major performance-flaws, because the
observed difference can be adequately explained by the known differences between these test setups.
It also gives credence to the idea that the impact of the differences between the benchmark and the EQueue
paper's testing is limited.

The other queues tested in the EQueue paper (B-Queue, MCRingBuffer, and FastForward Queue) reportedly took
around $\ns{112}$ per operation.
In my measurements, these queues all performed similarly to EQueue at around $\ns{19}$ per enqueue operation
and $\approx \ns{26}$ per dequeue operation.
This presents a major difference in results between my measurements and those of the EQueue paper.
The reasons for this difference are unclear.
They could for example be explained by differences in parameters for the other queues or perhaps by hardware
differences that specifically favor EQueue on the hardware used in that paper.
However, more research would be needed to determine with confidence the reasons for this discrepancy.

It can also be observed that all queues, except for Lamport queue, show a $\approx 50\%$ better dequeue
operation time when not including failed operations in the measurements.
This means that a significant portion of the dequeue operations of these queues is failing in this benchmark.
I suspect that this happens during the wait time between bursts and thus would not lead to decreased
application performance in a real world scenario.

Overall, it can be observed that all queues perform similarly in this benchmark, except for Lamport queue
which performs better than the other queues on the dequeue operations.
However, it is not clear whether this would lead to a real world performance advantage for Lamport queue.

\subsubsection{Bursty 65k-2k}
The results from the four variations of this benchmark can be seen in \autoref{fig:results-bursty-65k-2k}.
As with the previous benchmark, the variant of this benchmark without jitter and while including failed
operations in the measurements is the closest to the test from the EQueue paper with burst size $2,048$ (see
\autoref{fig:results-bursty-65k-2k-b}).

According to the EQueue paper, at a burst size of $2,048$, all queues tested there (EQueue, B-Queue,
MCRingBuffer, and FastForward Queue) took about $48$ cycles, which is about \ns{29}.
In my measurements, all queues except for EQueue perform similarly to each other at around \ns{10}-\ns{20}.
This difference to the EQueue paper is about what we would expect, since it lines up with the one observed
for EQueue in the previous benchmark and can therefore also be explained by the known differences between the
benchmarks.
However, at $\approx \ns{35}$ per enqueue operation, EQueue performs significantly worse than the other
queues in my measurements.
When not measuring failed operations, this difference can not be seen, which means that it appears due to
failed enqueue operations.
On its own this could be explained by there being less than $2,048$ free slots in the queue, leading to some
of the last elements from each burst not immediately fitting into the queue thus leading to failed enqueue operations.
However then, this effect would also be expected to be more pronounced with a higher burst size, which can
not be observed in the previous benchmark.
It is thus not clear why there is a higher number of failed operation for EQueue here, than there is for the
other queues.
Lastly, MCRingBuffer also seems to experience an increased number of failed dequeue operations over the other queues.
It is again not clear why this is the case, as successful queue operations of MCRingBuffer are similarly fast
to the other queues.

Overall, it can be said that the queues all perform relatively similar in this benchmark with the enqueue
operation of EQueue as an outlier and the dequeue operation of MCRingBuffer as another, smaller, outlier.

\subsubsection{Basic 65k}
For this benchmark the results of the four variations can be seen in \autoref{fig:results-basic-65k}.
In the case for the results of the variation without jitter and while measuring failed operations,
it can be observed that the most of the results are almost identical to those of the Bursty 65k-16k benchmark,
with the enqueue operation time of B-Queue, EQueue, and Lamport Queue being outliers,
which are approximately one order of magnitude worse than in the Bursty 65k-16k benchmark.
Comparing those results to the variation with neither jitter, nor measuring of failed operations,
this increase in operation time can only be seen for Lamport queue and only to a smaller degree.
This means that while EQueue and B-Queue are only outliers due to failed operations,
for Lamport queue the duration of successful enqueue operations is actually higher than in the other benchmarks.
This could mean that that some difference in timing leads to EQueue, B-Queue, and Lamport queue filling up
early in the benchmark, leading to failed enqueue operations later on because the queues are full.
As for the increase in duration for Lamport queue's enqueue operations, which isn't present for the other queues,
this could be due to cache optimisations, as Lamport queue is the first concurrent lock free queue to be presented,
with the other queues all focusing on optimisations over existing queues which Lamport queue does not
have\cite{Lamport}.

This benchmark is also, to a degree, similar to the testing done in the FastForward paper, where FastForward
queue and Lamport queue were compared.
However, in the FastForward paper three cores were setup to forward in a circle with each pair of cores being
connected by a queue, which presents a significant difference to this benchmark\cite{FastForward}.
Futhermore, in this benchmark the queues are of much larger size, which is another difference to the
FastForward paper's testing.
Despite those differences, my measurements confirm that Lamport queue does indeed perform much worse than
FastForward queue, both when measuring failed operations and when not doing so, as was found in the FastForward paper.
However, the performance is not identical to that measured in the FastForward paper, as could reasonably be
expected given the differences in testing setups.

This benchmark is also the most similar one, out of the ones included, to the testing in the B-Queue paper
with 1 concurrent queue.
However, the testing in the B-Queue paper used much smaller queues, so some differences could be expected due to that.
The B-Queue testing showed operation times of around \ns{8} for B-Queue, MCRingBuffer, and FastForward queue.
My measurements also show very similar operation times for these three queues, however,
the absolute operation times are at around \ns{10} for dequeue operations and at around \ns{19} for enqueue operations.
Since the B-Queue measurements were obtained by measuring the time, performing all queue operation, and
measuring the time again, for them the maximum operation time between the enqueue operation time and the
dequeue operation time is what they measured as the general operation time.
Applying this to my measurements leads to an effective general operation time of $\approx\ns{19}$, in
contrast to the $\approx\ns{8}$ from the B-Queue paper, meaning the B-Queue measurements are lower than any
result from my measurements or the measurements from the EQueue paper for any of the three queues\cite{B-Queue,EQueue}.
This suggests that the absolute values measured in the B-Queue paper are not comparable to the results from
my measurements or those from the EQueue paper.
It is not clear what causes this difference, however, it could be due to hardware differences, as both the
EQueue paper and I used significantly newer CPUs for the testing than the B-Queue paper did.

Overall, these results suggest that in scenarios with low fluctuation of the enqueue rate, the performance of
FastForward queue, FastFlow queue and MCRingBuffer is best out of the queues tested here.
