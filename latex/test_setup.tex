\section{Test Setup}
\begin{itemize}
    \item reproducibility (what did I mean by that?)
\end{itemize}

\subsection{System Under Test}
\paragraph{Hardware}
\begin{itemize}
    \item SuperMicro SYS-521C-NR
    \item 1x Intel Xeon Gold 6548N
        \begin{itemize}
            \item 32 Cores
            \item 2.8 GHz Base Frequency
            \item 1 NUMA Node
            \item L1 Cache 80KB (per core)
            \item L2 Cache 2MB (per core)
            \item L3 Cache 60MB (shared)
        \end{itemize}
    \item 128GB Memory
        \begin{itemize}
            \item PN: HMCG78MEBRA115N
            \item 4800MT/s
            \item 8x16GB
            \item DDR5
        \end{itemize}
    \item all tests at Base Freq for stability
\end{itemize}

\paragraph{Software}
\begin{itemize}
    \item Ubuntu 24.04.2 LTS
    \item Linux Kernel 6.8.0
    \item gcc 14.2.0
\end{itemize}

\subsection{Benchmarks}
\todo{cycles vs. ns}
\begin{itemize}
    \item will start by explaining what parameters there are per queue
    \item will then for each scenario explain which values were chosen and why + config file for it
\end{itemize}

\paragraph{BQueue}
\begin{itemize}
    \item \texttt{size} - Size of the queue
    \item \texttt{batch\_size} - BQueue does Prod/Con Batching so it doesn't have to check everytime to avoid
        running into the producer/consumer
    \item \texttt{batch\_increment} - BQueue dynamically adjusts the dequeue batch size, this value
        determines how much larger the candidate for the next batch size should be than the last batch size
    \item \texttt{wait\_time} - when determining the new dequeue batch size, bqueue waits a certain amount of
        time when a new candidate is chosen
\end{itemize}

\paragraph{EQueue}
\begin{itemize}
    \item \texttt{min\_size}, \texttt{max\_size} - EQueue dynamically adjusts the size of the queue to
        optimise cache access times, these values determine the bounds of that
    \item \texttt{initial\_size} - The initial size of the Queue before dynamic adjustments
    \item \texttt{wait\_time} - When determining the new enqueue batch size, equeue waits a certain amount of
        time when a new candidate is chosen
\end{itemize}

\paragraph{FastFlow Queue}
\begin{itemize}
    \item FFLWQ uses a linked list of buffers as queue \todo{should be true, but double check for details}
    \item \texttt{bucket\_size} - Size of the buffers (``buckets'')
    \item \texttt{max\_bucket\_count} - The maximum number of buckets ffq will allocate
\end{itemize}

\paragraph{MCRingBuffer}
\begin{itemize}
    \item \texttt{size} - size of the queue
    \item \texttt{batch\_size} - MCRB does Con/prod batching, this is the size of those batches
\end{itemize}

\paragraph{FastForward Queue}
\begin{itemize}
    \item \texttt{size} - size of the queue
    \item \texttt{adjust\_slip\_interval} - FFWDQ tries to delay dequeues via waiting so that over one
        cacheline of data is in the queue (the slip), this determines every how many dequeues this slip is
        reestablished \todo{check whether the slip is currently set correctly - because of interaction with
        bursty traffic}
\end{itemize}

\paragraph{Lamport Queue}
\begin{itemize}
    \item only \texttt{size} param for queue size
\end{itemize}

\subsubsection{Bursty 65k}
\begin{itemize}
    \item \texttt{equeue\_repro.toml}
    \item based on EQueue benchmark
    \item Message Count $1,000,000$
    \item Enqueue / Dequeue Rate $1,000,000$ \todo{why this number?}
    \item => 1Mpps for 1 sec
    \item enqueue burst size of $16,384$ (should according to EQueue paper show large difference; shouldn't
        run into possible weird issues around half queue size)
    \item dequeue constant wait
    \item Queue Size $65,536$
        \begin{itemize}
            \item max queue size of equeue is 65k
            \item size of other queues not mentioned
            \item keeping max queue size same for all queue should make for comparable results
        \end{itemize}
\end{itemize}

\todo{put all of these toml code blocks into a single figure instead?}
\begin{minted}{toml}
[BQueue]
size = 65536
batch_size = 4096
batch_increment = 2048
wait_time = 358
\end{minted}
\paragraph{BQueue}
\begin{itemize}
    \item \texttt{batch\_size} set to a 16th of the queue size as specified by BQueue source code
    \item \texttt{batch\_increment} set to half the \texttt{batch\_size} as specified by BQueue source code
    \item BQueue uses 1000 Cycles by default, \SI{358}{\nano\second} is 1000 Cycles on my hardware
\end{itemize}

\begin{minted}{toml}
[EQueue]
initial_size = 8192
min_size = 256
max_size = 65536
wait_time = 358
\end{minted}
\paragraph{EQueue}
\begin{itemize}
    \item \texttt{initial\_size} wasn't specified, tuned it
    \item \texttt{min\_size}, \texttt{max\_size} explicitly specified in EQueue paper
    \item \texttt{wait\_time} specified in paper and source code as 1000 Cycles, \SI{358}{\nano\second} is 1000 Cycles on my hardware
\end{itemize}

\begin{minted}{toml}
[FastFlow]
bucket_size = 4096
max_bucket_count = 16
\end{minted}
\paragraph{FastFlow Queue}
\begin{itemize}
    \item \texttt{bucket\_size} wasn't specified in paper, tuned
    \item \texttt{max\_bucket\_count} selected to match max queue size 0f 65ki
\end{itemize}

\begin{minted}{toml}
[MCRingBuffer]
size = 65536
batch_size = 8000
\end{minted}
\paragraph{MCRingBuffer}
\begin{itemize}
    \item \texttt{batch\_size} tuned - needs to be divisor of message count, will deadlock otherwise
\end{itemize}

\begin{minted}{toml}
[FastForward]
size = 65536
adjust_slip_interval = 64
\end{minted}
\paragraph{FastForward Queue}
\begin{itemize}
    \item \texttt{adjust\_slip\_interval} taken from paper and verfied by tuning
\end{itemize}

\begin{minted}{toml}
[Lamport]
size = 65536
\end{minted}
\paragraph{Lamport Queue}
\begin{itemize}
    \item no params but size
\end{itemize}

\todo{bullet points}
