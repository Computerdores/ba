\section{Test Setup}
\subsection{System Under Test}
\subsubsection{Hardware}
\begin{itemize}
    \item SuperMicro SYS-521C-NR
    \item 1x Intel Xeon Gold 6548N
        \begin{itemize}
            \item 32 Cores
            \item 2.8 GHz Base Frequency
            \item 1 NUMA Node
            \item L1 Cache 80KB (per core)
            \item L2 Cache 2MB (per core)
            \item L3 Cache 60MB (shared)
        \end{itemize}
    \item 128GB Memory
        \begin{itemize}
            \item PN: HMCG78MEBRA115N
            \item 4800MT/s
            \item 8x16GB
            \item DDR5
        \end{itemize}
    \item all tests at Base Freq for stability (set via \texttt{cpupower} util)
\end{itemize}

\subsubsection{Software}
\begin{itemize}
    \item Ubuntu 24.04.2 LTS
    \item Linux Kernel 6.8.0
    \item gcc 14.2.0
    \item core isolation using kernel params (e.g. \texttt{isolcpus=2,3 nohz\_full=2,3})
        \footnote{The EQ, BQ and FFWDQ papers don't mention whether or not, they did this.}
\end{itemize}

\subsection{Benchmarks}
\begin{itemize}
    \item I am using \texttt{RDTSC} to measure time
    \item does not meaure time directly; measures clock ticks
    \item incremented at constant rate of CPU base frequency if CPU has invariant TSC
    \item so to convert \texttt{RDTSC} diff to seconds divide by base freq

    \item will start by explaining what parameters there are per queue
    \item will then for each scenario explain which values were chosen and why + config file for it
\end{itemize}

\subsubsection{Parameters}
\paragraph{BQueue}
\begin{itemize}
    \item \texttt{size} - Size of the queue
    \item \texttt{batch\_size} - BQueue does Prod/Con Batching so it doesn't have to check everytime to avoid
        running into the producer/consumer
    \item \texttt{batch\_increment} - BQueue dynamically adjusts the dequeue batch size, this value
        determines how much larger the candidate for the next batch size should be than the last batch size
    \item \texttt{wait\_time} - when determining the new dequeue batch size, bqueue waits a certain amount of
        time when a new candidate is chosen
\end{itemize}

\paragraph{EQueue}
\begin{itemize}
    \item \texttt{min\_size}, \texttt{max\_size} - EQueue dynamically adjusts the size of the queue to
        optimise cache access times, these values determine the bounds of that
    \item \texttt{initial\_size} - The initial size of the Queue before dynamic adjustments
    \item \texttt{wait\_time} - When determining the new enqueue batch size, equeue waits a certain amount of
        time when a new candidate is chosen
\end{itemize}

\paragraph{FastFlow Queue}
\begin{itemize}
    \item FFLWQ uses a linked list of buffers as queue
    \item \texttt{bucket\_size} - Size of the buffers (``buckets'')
    \item \texttt{max\_bucket\_count} - The maximum number of buckets ffq will allocate
\end{itemize}

\paragraph{MCRingBuffer}
\begin{itemize}
    \item \texttt{size} - size of the queue
    \item \texttt{batch\_size} - MCRB does Con/prod batching, this is the size of those batches
\end{itemize}

\paragraph{FastForward Queue}
\begin{itemize}
    \item \texttt{size} - size of the queue
    \item \texttt{adjust\_slip\_interval} - FFWDQ tries to delay dequeues via waiting so that over one
        cacheline of data is in the queue (the slip), this determines every how many dequeues this slip is
        reestablished
\end{itemize}

\paragraph{Lamport Queue}
\begin{itemize}
    \item only \texttt{size} param for queue size
\end{itemize}

\subsubsection{Bursty 65k-16k}
\label{sec:bench-bursty-65k-16k}
\begin{itemize}
    \item \texttt{equeue\_repro\_16k.toml}
    \item based on EQueue benchmark
    \item Message Count $1,000,000$
    \item Enqueue / Dequeue Rate $1,000,000$
        \begin{itemize}
            \item differs from EQueue (they did ~20M)
            \item should test robustness of existing results \todo{is that a good enough reason?}
        \end{itemize}
    \item enqueue burst size of $16,384$
        \begin{itemize}
            \item burst size selected because it should show large difference (according to EQueue paper)
            \item concrete decision of 16k vs 32k arbitrary (both showed large difference between queues in
                equeue paper)
            \item bursty waiter differs slightly from equeue paper
            \item instead of same wait time varies to adjust for operation time
            \item based on idea of network application so enqueue rate should be the same across all queues
                (fixed wait time would mean varying rate due to varying operation times)
        \end{itemize}
    \item dequeue constant wait (most similar to equeue paper)
        \begin{itemize}
            \item uses constantWait module for waiter
            \item same as specified in EQueue paper
        \end{itemize}
    \item Queue Size $65,536$
        \begin{itemize}
            \item max queue size of equeue is 65k
            \item size of other queues not mentioned
            \item keeping max queue size same for all queue should make for comparable results
        \end{itemize}
\end{itemize}

\begin{figure}
    \centering
    \begin{minted}{toml}
        [BQueue]
        size = 65536
        batch_size = 4096
        batch_increment = 2048
        wait_time = 358

        [EQueue]
        initial_size = 8192
        min_size = 256
        max_size = 65536
        wait_time = 358

        [FastFlow]
        bucket_size = 4096
        max_bucket_count = 16

        [MCRingBuffer]
        size = 65536
        batch_size = 8000

        [FastForward]
        size = 65536
        adjust_slip_interval = 64

        [Lamport]
        size = 65536
    \end{minted}
    \caption{The Queue parameters for the Bursty 65k benchmark.}
    \label{fig:toml-bursty-65k}
\end{figure}

\paragraph{BQueue}
\begin{itemize}
    \item \texttt{batch\_size} set to a 16th of the queue size as specified by BQueue source code
    \item \texttt{batch\_increment} set to half the \texttt{batch\_size} as specified by BQueue source code
    \item BQueue uses 1000 Cycles by default, \SI{358}{\nano\second} is 1000 Cycles on my hardware
    \item See \autoref{fig:toml-bursty-65k} for the concrete values of this and the next paragraphs
\end{itemize}

\paragraph{EQueue}
\begin{itemize}
    \item \texttt{initial\_size} wasn't specified, tuned it
    \item \texttt{min\_size}, \texttt{max\_size} explicitly specified in EQueue paper
    \item \texttt{wait\_time} specified in paper and source code as 1000 Cycles, \SI{358}{\nano\second} is
        1000 Cycles on my hardware
\end{itemize}

\paragraph{FastFlow Queue}
\begin{itemize}
    \item \texttt{bucket\_size} wasn't specified in paper, tuned
    \item \texttt{max\_bucket\_count} selected to match max queue size 0f 65ki
\end{itemize}

\paragraph{MCRingBuffer}
\begin{itemize}
    \item \texttt{batch\_size} tuned - needs to be divisor of message count, will deadlock otherwise
\end{itemize}

\paragraph{FastForward Queue}
\begin{itemize}
    \item \texttt{adjust\_slip\_interval} taken from paper and verfied by tuning
\end{itemize}

\paragraph{Lamport Queue}
\begin{itemize}
    \item no params but size
\end{itemize}

\subsubsection{Bursty 65k-2k}
\begin{itemize}
    \item \texttt{equeue\_repro\_2k.toml}
    \item same as the Bursty 65k-16k, but burst size is 2k instead of 16k
    \item EQueue tested this (with bursty) as well, showed negligible difference between queues
\end{itemize}
\todo{maybe merge this with the Bursty 65k-16k subsubsection?}

\subsubsection{Basic 65k}
\begin{itemize}
    \item same as Bursty 65k-16k (See \autoref{sec:bench-bursty-65k-16k})
    \item instead of bursty enqueue, enqueue with constant rate
    \item via ConstantRate waiter instead of bursty waiter
        \begin{itemize}
            \item EQueue does not test a scenario like this but should be interesting
            \item more similar to bqueue testing where they had constant waits between enqueues
        \end{itemize}
\end{itemize}
