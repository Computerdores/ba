\section{Introduction}
\label{sec:introduction}
Parallelisation of applications is an important technique for reaching performance goals.
A common example of such a utilization of multi-core architectures, are network applications making use
of pipeline parallelism\cite{Upadhyaya2007,WangCheng2009}.
This allows utilising multiple threads by decomposing these network applications into sequential tasks that can be
parallelised\cite{WangCheng2009}.
In such applications, FIFO queues play an important role as communication channels between these sequential
tasks, which can also often be located on different CPU cores.
However, the performance of these communication channels can be critical to effectively making use of
parallelisation, ``take [a] 10 Gbps network as an example, a minimal 64-byte Ethernet frame must be processed
in amortized time of 67 ns, which is about one DRAM access time''\cite{B-Queue}.

Because of this there are many different SPSC FIFO queue designs aiming to improve, relative to previous
iterations, the time spent per enqueue/dequeue operation\cite{B-Queue,EQueue,FastForward,FastFlowGithub,MCRingBuffer}.
However, choosing one of these implementations can be diffcult for several reasons.
Each paper concludes that its proposed design surpasses the performance of existing designs in some
respect and presents test results to support the performance improvement in that respect.
This means that these benchmarks are very different from paper to paper, making it hard to compare the results.
Furthermore, there also don't seem to be any papers comparing many different queues via the same tests (See
\autoref{app:repro-search-queries}).
The papers also rarely provide source code for their new queue designs or, if they do, the source code may
not be available anymore due to its external hosting having become unavailable\footnote{Examples of externally
    hosted source code becoming unavailable include the original distributions of the source code of the EQueue
and B-Queue papers (See \autoref{equeue-bqueue-links}).}.
All of this makes a direct replication hard, if not impossible\cite{sep-scientific-reproducibility}.

With the aim of addressing these problems, I built a benchmarking framework and used it to benchmark six
queues: B-Queue, EQueue, FastForward Queue, FastFlow Queue, Lamport Queue and MCRingBuffer\footnote{These six
    queues were selected due to being compared by the EQueue and FastForward papers, with EQueue initially being
selected due to having been considered for use in a parallelised network application.}
\cite{B-Queue,EQueue,FastForward,FastFlowGithub,Lamport,MCRingBuffer}.
This Framework should make it easier for future authors to test queue designs in a way that is replicable and
comparable, and makes it easy to include parameter values like queue size, batching size, etc. with the
source code by reading these values from a config file that can be stored alongside it.
Afterwards, I compare the results from the benchmarks to those of the EQueue, B-Queue and FastForward papers
in order to see how closely their results are replicated.
These benchmarks were designed to be most similar to those of the EQueue paper, as that paper compares the
largest number of different queues out of those three papers.

In order to make my results as replicable as possible, I released the full source code on Github (see
\href{https://github.com/Computerdores/ba}{Computerdores/ba}) and also embedded it into the PDF to prevent
link rot (see \autoref{app:source-code-archive}).
Furthermore, I provide and explain my choice of all parameters of the tests and queues and also detail the
considerations that were made during the implementation of the queues.
Due to the full source code of all of these queues being released, all with an identical interface, it will
also be easier for implementors to test different queues in their applications.

In the rest of this thesis I first outline the architecture of the benchmarking framework that was developed
and certain technical details of it.
Afterwards, I describe the considerations made in the queue implementions, followed by an explanation of the
system under test and the different benchmarks that were performed.
Then, I analyse the results of the benchmarks, and finally I present my conclusion.
