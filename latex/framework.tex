\section{Framework}
\todo{more details?}
\subsection{Goals}
\todo{TODO: not sure about this subsection title}
\begin{itemize}
    \item want to avoid duplicating the entire benchmark code for every queue/variation in a part of the benchmark
    \item duplicating the code would allow the compiler to optimise better
    \item don't want to give up on that performance advantage (I would expect that to be present in real applications)
    \item EQueue: parts of queue impl in benchmark code => I want full separation
    \item should allow new queues or benchmark variations with minimal effort
    \item full source code available on GitHub (See \href{https://github.com/Computerdores/ba}{Computerdores/ba})
\end{itemize}

\subsection{Architecture}
\begin{figure}
    \begin{minted}{cpp}
/* ... */
template <IsQueueWith<u64> Q, IsMeasurerPair M, IsWaiterPair W, bool MEASURE_FAILED = false>
class Runner {
  public:
    template <typename P>
    explicit Runner(Q* queue, const P& params)
        : _queue(queue), _measurer(M(params.msg_count)), _waiter(W(params)),
          _msg_count(params.msg_count) {}

    void run() { /* ... */ }

    void write_results(std::ostream* out) { /* ... */ }

  private:
    Q* _queue;
    M _measurer;
    W _waiter;
    usize _msg_count;
    volatile bool _start = false;
    CACHE_ALIGNED u64 _data_sink;

    void producer() {
        // wait for start
        while (!_start) { }
        // do test
        _waiter.tx.start();
        for (usize i = 0; i < _msg_count; i++) {
            _measurer.tx.pre();
            auto data = get_timestamp();
            while (!_queue->enqueue(data)) {
                if constexpr (!MEASURE_FAILED) _measurer.tx.pre();
            }
            _measurer.tx.post();
            _waiter.tx.wait();
        }
    }

    void consumer() {
        // wait for start
        while (!_start) { }
        // do test
        _waiter.rx.start();
        for (usize i = 0; i < _msg_count; i++) {
            _measurer.rx.pre();
            std::optional<u64> data = std::nullopt;
            while (!((data = _queue->dequeue()))) {
                if constexpr (!MEASURE_FAILED) _measurer.rx.pre();
            }
            _data_sink += *data;
            _measurer.rx.post();
            _waiter.rx.wait();
        }
    }
};
    \end{minted}
    \caption{The implementation of the runner class in C++, with some parts ommited.}
    \label{fig:runner-class-code}
\end{figure}
\begin{itemize}
    \item Runner class as a skeleton benchmark (See \autoref{fig:runner-class-code})
    \item populated with templated components
    \item components defined in headers
    \item => get inlined into the runner class (verified in Ghidra)
    \item => almost identical result of compilation to duplicating the runner class and copying the component
        code into it
    \item => no performance difference, but easier to work with
    \item runner class has three types of components:
    \item waiter
        \begin{itemize}
            \item causes a delay between operation
            \item used to emulate different enqueue/dequeue patterns
            \item has a \texttt{start} method to mark the beginning of the benchmark
            \item has a \texttt{wait} method to trigger a wait time according to the waiter's semantics
        \end{itemize}
    \item measurer
        \begin{itemize}
            \item measures time between start and end of operation
            \item can be used to implement differnt ways of measuring time
            \item can also be varied to measure other aspects (cnt failed ops, cache stats, etc.) there
            \item \texttt{pre} method gets called to record timestamp before operation (overwrites last
                timestamp if called back to back)
            \item \texttt{post} method gets called to record timestamp after operation
        \end{itemize}
    \item queue
        \begin{itemize}
            \item the data structure to be benchmarked
            \item implements consistent interface with non-blocking \texttt{enqueue} and \texttt{dequeue}
                that indicate success via their return values
        \end{itemize}
\end{itemize}

\subsubsection{Waiters}
\begin{itemize}
    \item Constant Wait
        \begin{itemize}
            \item waits a constant amount of time between operations
            \item used to emulate operations (of input / output) that take a very consistent amount of time
        \end{itemize}
    \item Constant Rate
        \begin{itemize}
            \item tries to maintain a constant time difference between starts of operations
            \item used to emulate input streams with a constant rate
        \end{itemize}
    \item e.g. if OP takes \ns{80} and wait time is \ns{200} then ConstantWait leads to a \ns{280} delay
        between starts of OPs and ConstantRate leads to a \ns{200} delay between starts of OPs because it
        only waits $\ns{200}-\ns{80}=\ns{120}$
    \item Bursty
        \begin{itemize}
            \item defers waits in order to have blocks of BURST\_SIZE elements with no waits
            \item used to emulate enqueue / dequeue behaviour that is bursty
            \item typical for network applications
            \item e.g. with BURST\_SIZE of 20, wait time of \ns{100}, and OP duration of \ns{20} -- 20
                elements will be enqueued followed by a wait of $20\cdot(\ns{100} - \ns{20}) = \ns{1600}$
        \end{itemize}
    \item Jitter can be enabled; adds pseudo-random variance to wait times (idea also present in FFWDQ paper)
\end{itemize}

\subsubsection{Measurers}
\begin{itemize}
    \item could vary timestamp source e.g. for different architectures
    \item could measure cache stats for the duration of the operation
    \item only one measurer implemented (=> used for all benchmarks here)
    \item uses \texttt{RDTSC} instruction to get clock count
    \item uses known base frequency to derive ns timestamps from clock count (epoch is not accounted for)
\end{itemize}
