\subsection{Benchmarks}
\todo{TODO: ``Querverweise'' -- ??}
In this subsection I will start by explaining the different parameters to be set for each queue.
While I will outline the purpose of each parameter, I will not provide an in-depth explanation of the inner
workings of each queue.
For detailed explanations, please refer to the original paper proposing the respective queues\footnote{Note,
    however, that the FastFlow paper does not give a detailed explanation of the queue, and only states that its
    ``core is based on efficient Single-Producer-Single-Consumer (SPSC) and Multiple-Producer-Multiple-Consumer
(MPMC) FIFO queues, which are implemented in a lock-free and wait-free fashion''\cite{FastFlow}.}.
I will then continue to outline the specifics of each benchmark, with one benchmark per subsection in the
order from the benchmark with the smallest difference between the performance of the queues to the benchmark
with the highest difference between the queues.

\subsubsection{Parameters}
\paragraph{BQueue}
BQueue implements producer and consumer batching to avoid having to read ahead in the buffer on every
operation in order to check if the operation will succeed \cite{B-Queue}.
BQueue also dynamically adjusts the dequeue batch size.
The queue is configured via the following values:
\begin{itemize}
    \item \texttt{size} -- The size of the queue.
    \item \texttt{batch\_size} -- This parameter determines the size of the producer and consumer batches.
    \item \texttt{batch\_increment} -- This value determines how much larger the candidate for the next
        dequeue batch size should be than the last used one.
    \item \texttt{wait\_time} -- While determining the new dequeue batch size, B-Queue waits a certain amount
        of time when a new candidate is chosen. This parameter determines the duration of each of those waits.
\end{itemize}

\paragraph{EQueue}
EQueue dynamically adjusts the size of the queue to optimise cache access times\cite{EQueue}. Furthermore,
EQueue also dynamically adjusts the dequeue batch size. The following parameters influence these behaviours:
\begin{itemize}
    \item \texttt{min\_size}, \texttt{max\_size} -- These values determine the valid range for the queue size. The
        underlying ring buffer will be allocated with a size of \texttt{max\_size}.
    \item \texttt{initial\_size} -- The initial size of the queue before the dynamic queue size adjustments.
    \item \texttt{wait\_time} -- When determining the new enqueue batch size, equeue waits a certain amount of
        time when a new candidate is chosen. This parameter determines the length of those waits.
\end{itemize}

\paragraph{FastFlow Queue}
FastFlow Queue uses a linked list of dynamically allocated buffers as the storage of the queue\cite{FastFlowGithub}.
The buffers are referred to as ``buckets.''
This behaviour is controlled by the following parameters:
\begin{itemize}
    \item \texttt{bucket\_size} -- This parameter sets the size of these buffers.
    \item \texttt{max\_bucket\_count} -- This parameter determines the maximum number of buckets that can be
        allocated by the queue.
\end{itemize}

\paragraph{MCRingBuffer}
MCRingBuffer implements producer and consumer batching\cite{MCRingBuffer}.
The queue is configured using the following values:
\begin{itemize}
    \item \texttt{size} -- The size of the queue.
    \item \texttt{batch\_size} -- The size of the producer and consumer batches.
\end{itemize}

\paragraph{FastForward Queue}
FastForward Queue delays dequeues via waiting to try to keep over one cacheline of data in the queue\cite{FastForward}.
The distance between head and tail that this creates is called the ``slip''.
The following parameters can be used to control FastForward Queue's behaviour:
\begin{itemize}
    \item \texttt{size} -- The size of the queue.
    \item \texttt{adjust\_slip\_interval} -- This parameter determines every how many dequeues the slip is adjusted.
\end{itemize}

\paragraph{Lamport Queue}
The only parameter of Lamport Queue is the \texttt{size} parameter, which controls its size\cite{Lamport}.

\subsubsection{Bursty 65k-16k / Bursty 65k-2k}
\label{sec:bench-bursty-65k-16k}

The Bursty 65k-16k and Bursty 65k-2k benchmarks are based on the tests performed in the EQueue paper,
specifically the tests with a burst size of $16,384$ and $2,048$ elements respectively.
The benchmark performs $1,000,000$ operations at a targeted rate for enqueue and dequeue of $1,000,000$
operations per second\footnote{This rate differs from the rate in the EQueue paper, where they targeted
    $\approx 20,000,000$ operations per second (specified as a wait time of $85\text{ cycles}\approx \ns{50}$).
\todo{TODO: Give reason here? If so, is ``should test robustness of existing results'' a good enough reason?}}.
All benchmarks (including Basic 65k) were repeated 100 times, with the operation times being averaged for
every repetition.

The benchmark uses the Bursty waiter component to simulate bursty enqueue patterns, similar to the EQueue paper.
The burst sizes of $16,384$ and $2,048$ were selected, because both were tested in the EQueue paper and showed
very different results.
The concrete selection of the burst size $16,384$ over the burst size $32,768$ was arbitrary due to the small
difference between the reported results.
It is further worth noting, that the implementation of the bursty enqueue pattern differes slightly from the
one used in the EQueue paper, because the Bursty waiter adjusts for the time the operation takes, as outlined
in \autoref{sec:waiter-implementations}.
The waiter for the consumer thread is a Constant Wait component, selected because it is closest to the
implementation from the EQueue paper.

The minimum and maximum size of EQueue is set to $256$ and $65,536$ respectively, as it also is in the EQueue paper.
The paper does not specifiy the queue sizes used for queues other than EQueue.
For this reason I chose $65,536$ as the (maximum) queue size for all other queues in order to have a
like-for-like comparison.
In the next paragraphs, I will outline the reasons for the remaining queue parameter values.
The configuration containing all of these values can be seen in \autoref{fig:toml-bursty-65k}.
The full configurations for both of these benchmarks can be found in \texttt{equeue\_repro\_16k.toml} and
\texttt{equeue\_repro\_2k.toml} respectively.

\begin{figure}
    \centering
    \begin{minted}{toml}
        [BQueue]
        size = 65536
        batch_size = 4096
        batch_increment = 2048
        wait_time = 358

        [EQueue]
        initial_size = 8192
        min_size = 256
        max_size = 65536
        wait_time = 358

        [FastFlow]
        bucket_size = 4096
        max_bucket_count = 16

        [MCRingBuffer]
        size = 65536
        batch_size = 8000

        [FastForward]
        size = 65536
        adjust_slip_interval = 64

        [Lamport]
        size = 65536
    \end{minted}
    \caption{The Queue parameters for the Bursty 65k benchmarks.}
    \label{fig:toml-bursty-65k}
\end{figure}


\paragraph{BQueue}
The \texttt{batch\_size} parameter was set to a sixteenth of the queue size as specified by the original
B-Queue source code.
In turn, the \texttt{batch\_increment} parameter was set to half of the \texttt{batch\_size} parameter as
also specified by the source code.
Lastly, the \texttt{wait\_time} parameter was set to \SI{358}{\nano\second} in accordance with the
$1,000\text{ cycles}$ value from the source code.

\paragraph{EQueue}
For the \texttt{initial\_size} value of EQueue, no value was specified in the paper, and it was thus tuned
from scratch\footnote{Note that all of the parameter tuning was done at significantly lower sample counts
than the final benchmarks, because of the unfeasibly large amount of time this would otherwise have taken.}.
The other unaccounted for parameter, \texttt{wait\_time}, was specified in paper and source code as 1000
Cycles and was thus set to the equivalent \SI{358}{\nano\second}.

\paragraph{FastFlow Queue}
For FastFlow queue there are two parameters; \texttt{bucket\_size} for which there is no specified default
value and which was thus tuned from scratch, and \texttt{max\_bucket\_count} which was entirely predetermined by
the bucket size and the targeted maximum size of the queue.

\paragraph{MCRingBuffer}
The \texttt{batch\_size} parameter was tuned -- with the constraint that it needed to be a divisor of the
number of enqueue / dequeue operations in order to avoid the aforementioned deadlock issue.

\paragraph{FastForward Queue}
The value for \texttt{adjust\_slip\_interval} was taken from the FastForward paper as is and then verfied via tuning.

\paragraph{Lamport Queue}
The Lamport queue has no parameters, but the size.

\subsubsection{Basic 65k}
The Basic 65k benchmark is identical to the Bursty benchmarks, except that the Constant Rate waiter was used
for the enqueue thread instead of the Bursty waiter.
The EQueue paper does not include a test with constant rate or constant wait enqueueing, but this
benchmark is more similar to the testing in B-Queue and FastForward queue than the Bursty benchmarks are.
The configuration file for this benchmark is \texttt{equeue\_repro\_16k.toml}.
